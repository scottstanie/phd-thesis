\hypertarget{disambiguating-terms-from-fractal-surfaces-gaussian-random-fields-and-power-laws}{%
\section{Disambiguating terms from fractal surfaces, Gaussian random
fields, and power
laws}
\label{disambiguating-terms-from-fractal-surfaces-gaussian-random-fields-and-power-laws}}

\textbf{Driving question}:

Is the fractal atmospheric surface (a type of spatially correlated
noise), which has a power-law power spectrum of

\[S(f) = (1/f)^{\beta}\] also a Gaussian random field?

The three related subjects are\ldots{} - Gaussian Fields - Fractal
surfaces - Power-law power spectra Can all three be present? Or are some
mutually exclusive?

\textbf{POSSIBLE SUMMARY ANSWER}

Yes, the surface can be 1. fractal, since the structure function (aka
semi-variogram) of the atmospheric delay has a power-law form (Hanssen,
Eq 4.7.10) \footnote{Hanssen, Ramon F. Radar interferometry: data
  interpretation and error analysis. Vol. 2. Springer Science \&
  Business Media, 2001.} :

\[D_s(\rho) \sim \rho^{5/3}\]

\begin{enumerate}
\def\labelenumi{\arabic{enumi}.}
\setcounter{enumi}{1}
\tightlist
\item
  A Gaussian field, since by (Cressie, 1993) Eq. 5.5.1 \footnote{Cressie,
    Noel. Statistics for spatial data. John Wiley \& Sons, 2015.},
\end{enumerate}

More generally, a fractional Brownian motion in \(R^d\) is a Gaussian
process \(Z(\cdot)\) characterized by a covariance function of the form
with

\[C(s, t) = \frac{1}{2}\left(s^{2H} + t^{2H} + |s-t|^{2H}\right)\]

and a variogram of the form

\[E[(Z(s + h) - Z(s))^2] \propto ||h||^{2H}\] So, for the Hanssen
structure fucntion, \(H=5/6\)

\begin{enumerate}
\def\labelenumi{\arabic{enumi}.}
\setcounter{enumi}{2}
\tightlist
\item
  A signal power law power spectrum \(S(k)\) of the form (Hanssen Eq.
  4.7.12)
\end{enumerate}

\[P(k) = P_0 \left(f/f_0\right)^{-\beta}\] where \(P_0\) is the power at
reference frequency \(f_0\) and \(\beta\) is called the \emph{spectral
index}. The structure function of a signal with this power spectrum can
be written, using the same \(\beta\), as
\[D_{\phi}(\rho) \propto \rho^{\beta-1}\] which means that if the
structure function exponent is \(5/3\), then \(\beta=8/3\) for the
spectrum slope.

My confusion earlier: thinking that the frequency content of the signal
was doing to dictate the amplitude distribution in space (or time for
1D)\ldots{} In the same way that noise can be white (a
frequency/spectrum description) and either uniform or Gaussian (a time
or space description), the random process can be a Gaussian process, but
have the same power-law exponent.

LAST QUESTIONs: - BIGGEST UNKNOWN: Hanssen says the structure
functions/SMV is \(\rho^{5/3}\), which means in never levels off, and he
says the covariance function doesn't exist\ldots{} - Does this imply
that it can't be a Gaussian field? - \textbf{ANSWER}: most places seem
to say that fractional brownian motion is still a Gaussian field
(Brownian sheet having covariance = min(s, t)) - \textbf{TODO} This
means it has stationary increments, but isn't 2nd order
stationary\ldots{} I thought other places said ``Gaussian random fields
are 2nd order stationary (and sometimes strictly stationary under some
condition)'' - or does out limit study area + ramp removal mean in
practice it levels off\ldots{} - if a random field is Gaussian, does
it's structure func/SMV have to be gaussian?\ldots{}. NO. thats related
to the Covariance FUNCTION\ldots{} aka, the covariance of the Gaussian
variables at a given distance apart! the further away, the different the
covariance is\ldots{} but the entire process is still only characterized
by it's mean (assumed 0), and covariance function (related to the
structure function) - SO, a power-law SMV is fine for a Gaussian
process. and it will have a power law power-spectrum by FT:
https://math.stackexchange.com/questions/2173780/computing-fourier-transform-of-power-law

\begin{itemize}
\tightlist
\item
  I thought that Gaussian processes had have Gaussian Fourier
  transforms\ldots{}

  \begin{itemize}
  \tightlist
  \item
    answer? the power spectrum relates to the FT of the covariance
    function of a (second order stationary process) (or, with different
    terms, ACF \textless{}-\textgreater{} PSD)
  \end{itemize}
\item
  so i think ``fractal'' descriptions are orthogonal to both ``Gaussian
  process'' and ``Power spectrum'' shape\ldots{} they might just be 3
  completely separate axes, where none imply others
\end{itemize}

\hypertarget{definitions}{%
\section{Definitions}\label{definitions}}

From Hanssen, 4.7 intro: \textgreater{} The behavior of atmospheric
signal in radar interferograms can be mathematically described using
several interrelated measures such as the power spectrum, the covariance
function, the structure function, and the fractal dimension.

\begin{quote}
The power spectrum is useful to recognize scaling properties of the data
or to distinguish different scaling regimes.
\end{quote}

TODO 1. Random field 1. Gaussian random field 1. Power spectrum
