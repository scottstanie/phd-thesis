
\chapter{ Permian Basin Background}
\label{CHAP:3}



%1. Oil and gas production from horizontal drilling and fracking
%2. Induced seismicity: historical in Texas
%3.  Why is it happening?
  % Basic mechannisms of induced. pore pressure, mohr circle
  % other methods to tie: spatio temporal linking. more plausible when there are many simultaneous factors
 % 4. recent stuff (figure from GRL)
 % Why hard to 

%1. different theories of causes 
%2. difficulty in the Delaware basin case due to density
%3. Difficulty of observing subsurface changes

\section{Shale Development and Induced Seismicity}
\label{sec:ch3-oil}

Texas has been a leading producer of oil and gas for over a century \citep{Frohlich2016HistoricalReviewInduced, TheAcademyofMedicine2017EnvironmentalCommunityImpacts}. It became the nation's largest producer of crude oil after the first successful vertical well was drilled south of Beaumont, TX in 1901. These ``conventional'' wells were the primary mode of production in multiple oil fields across the state. It wasn't until the early 2000s that advances horizontal drilling and hydraulic fracturing (also known as fracking, Figure \ref{fig:oil-drilling}) opened up vast new shale resources which were previously unworkable \citep{Waters2006Spe103202Ms}. 
For example, the Wolfcamp shale in Texas' Permian Basin is the largest continuous oil field that has ever been discovered in the United States, containing 20 billion barrels of oil and 16 trillion cubic feet of gas \citep{Gaswirth2016AssessmentUndiscoveredContinuous}. While areas of the Wolfcamp shale in the Midland Basin have been traditionally developed using vertical wells, the ability to extend subsurface drilling horizontally (Figure \ref{fig:ch3-oil-drilling}b) and increase production using enhanced oil recovery (EOR, Figure \ref{fig:ch3-oil-drilling}d)
allowed many new areas to be economically viable for oil and gas production (Figure \ref{fig:ch3-drilling-pads}).


%Figure \ref{fig:oil-drilling} shows a simplified diagram of the operation techniques of horizontal drilling, hydraulic fracturing, wastewater disposal, and enhanced oil recovery.
%These shale resources are also known as ``tight oil` and ``shale gas'', and may collectively be referred to as ``unconventional''. 


\begin{figure}
	\centering
%	\includegraphics[width=0.6\linewidth]{figures/chapter3-permian/oil-page.pdf}
	\includegraphics[width=\linewidth]{figures/chapter3-permian/oil-page-square.pdf}
	\caption[Diagram of unconventional oil production operations]{
		Simplified diagrams of oil-field operations. Arrows show the directions of fluid being injected or withdrawn. Arrow color indicates the contents of the fluid: black (oil, gas, and wastewater), yellow (oil and gas), and blue (wastewater). 
		(a) In a hydraulic fracturing operation, fluids are injected at high pressure into a production well, causing fractures in the surrounding rock that increase permeability. Increased permeability allows the extraction of oil or gas from a larger region. Following the hydraulic fracturing of a well, the well goes into production (b). 
		(c) Production wells extract oil and gas, and as a byproduct, salt water (commonly called ``produced water'' or ``wastewater''), which is injected to a different subsurface formation at a disposal well. 
		(d) Enhanced oil recovery (EOR), an alternative to wastewater disposal, involves injecting the water back into the formation holding the oil and gas to sweep oil and gas toward the production well.
		(Figure adapted from \citep{Rubinstein2015MythsFactsWastewater})
	}
	\label{fig:ch3-oil-drilling}
\end{figure}

%This report uses the term “shale” to describe organic rich formations containing natural gas (shale gas) and/or oil (tight oil) that require multiple hydraulic fractures, usually created from long wells drilled horizontally, to produce hydrocarbons profitablyThe term “tight oil” may include hybrid formations containing oil that has migrated into very tight rock. Many experts refer to shale gas and tight oil resources collectively as “unconventional.” \cite{TheAcademyofMedicine2017EnvironmentalCommunityImpacts}
%lthough these resources have been known to exist for decades, rapid expansion of oil and gas production from shale formations was made possible by the innovative combined use of two technologies—hydraulic fracturing (referred to colloquially as “fracking”) and horizontal drilling.


\begin{figure}
	\centering
	\includegraphics[width=\linewidth]{figures/chapter3-permian/permian-images.pdf}
	\caption[Permian Basin drilling pads]{
		(a) Aerial view of drilling pads throughout the Permian Basin
		(b) Drilling rig set up on one pad
		(Source: XTO Energy)
		(c) Water condensate pit used to store fresh water condensed from natural gas or other flowback fluids (photo source: Benjamin Lowy)
	}
	\label{fig:ch3-drilling-pads}
\end{figure}


Despite the economic benefits that the new production technologies provided for Texas, concerns have been raised about possible environmental consequences to shale development \citep{TheAcademyofMedicine2017EnvironmentalCommunityImpacts, Scanlon2020WillWaterIssues}. 
%including injection of chemicals underground, water usage in semiarid areas and potential water contamination, and seismic activity
%For example, more than 30 billion bbl of wastewater have been disposed throughout the Permian Basin \cite{Lemons2019SpatiotemporalStratigraphicTrends} and future Wolfcamp shale development is expected to generate $\sim$300 billion bbl of produced water \cite{Smye2021VariationsVerticalStress,  Scanlon2020WillWaterIssues}.
%Here we briefly review the  seismic activity induced from oil and gas production.
One concern is the triggering of seismic activity, as it has been recognized that injection or withdrawal of fluids from the subsurface can induce earthquakes along existing faults \citep{Ellsworth2013InjectionInducedEarthquakes, Simpson1988TwoTypesReservoir}.
Note that induced earthquakes are not limited to oil and gas operations \citep{Grigoli2017CurrentChallengesMonitoring, Foulger2018GlobalReviewHuman, Baan2017HumanInducedSeismicity}; they have also been associated with geothermal energy development \citep{Deichmann2009EarthquakesInducedStimulation}, mining operations \citep{Hasegawa1989InducedSeismicityMines}, water impoundment in reservoirs \citep{Talwani1997NatureReservoirInduced}, and CO$_2$ sequestration \citep{Gan2013GasInjectionMay}.



%2. Induced seismicity: historical in Texas
%3.  Why is it happening?
% Basic mechannisms of induced. pore pressure, mohr circle
% other methods to tie: spatio temporal linking. more plausible when there are many simultaneous factors



% 
%During this time period, an increase in low magnitude earthquakes was observed. While West Texas had few seismic monitoring stations set up, one array near Lajitas, TX had been operating since 2000 \citep{Frohlich2019OnsetCauseIncreased}.


\begin{figure}
	\centering
	%	\includegraphics[width=\linewidth]{figures/chapter3-permian/mohr-circles.pdf}
	\includegraphics[width=\linewidth]{figures/chapter3-permian/mohr-circles-toponly.pdf}
	\includegraphics[width=\linewidth]{figures/chapter3-permian/injection-ellsworth.pdf}
	\caption[Effects of pore pressure perturbations and poroelastic stress changes on fault failure]{
		Effects of pore pressure perturbations and poroelastic stress changes on fault failure. Solid curves represent the initial stress state, and dashed curves represent the perturbed stress state. (a) Increased pore pressure reduces normal stress on the fault plane, moving the fault closer to the Coulomb failure criterion. (b) Poroelastic stresses increase differential stress. For both (a) and (b), pore pressure perturbations and stress changes, as well as the relative magnitude of changes, depend on parameters including time, distance, injection rate, diffusivity, and poroelastic parameters.
		(Bottom) Schematic diagram for each mechanism causing injection-induced earthquakes.
		(Top from \cite{Keranen2018InducedSeismicity}, bottom from \cite{Ellsworth2013InjectionInducedEarthquakes})
	}
	\label{fig:mohr-circles}
\end{figure}

One conceptual model for triggering earthquakes uses the Mohr-Coulomb failure criterion \citep{Hubbert1959RoleFluidPressure}.
The critical shear stress $\tau_{critical}$ required to promote fault slippage can be written as
\begin{equation}
\tau_{critical} = \tau_0 + \mu (\sigma_n - P)
\end{equation}
where $\tau_0$ is the cohesive strength of the sliding surface (often negligible), $\mu$ is the coefficient of friction, $\sigma_n$ is the normal stress, $P$ is the pore pressure \citep{Nicholson1990EarthquakeHazardAssociated, Ellsworth2013InjectionInducedEarthquakes}. Intuitively, increasing the shear stress or decreasing the normal stress ``unclamps'' the fault and encourages failure \citep{Shearer2019IntroductionSeismology}. Since increasing pore pressure lowers the effective normal stress, fluid injection can move critically stressed faults to failure and cause earthquakes (Figure \ref{fig:mohr-circles}a).
Alternatively, poroelastic effects from injection can change the loading conditions on a fault without a direct hydraulic connection, which can cause fault failure through increased differential stress (Figure \ref{fig:mohr-circles}b).

%This was determined to be the cause of many earthquakes in Oklahoma in 2014-2016 (cite). However, other causes may play a role in the Permian Basin earthquakes, including poroelastic effects that may be a factor 10s of kilometers away from injection, even without direct hydraulic connections \citep{Skoumal2020InducedSeismicityDelaware}


%Pressure from wastewater injection may play a role in certain areas (find that citation). This increasing of subsurface pore pressure could cause surface uplift in certain cases, and an example of this was reported on a single injection well (and single CO2 injection?) by \cite{Kim2018AssociationLocalizedGeohazards}.s



%Following \cite{Dahm2012RecommendationDiscriminationHuman}, we can classify these discrimination methodsin three main families: physics-based methods, statistics-based methods, and source parameters-basedmethods




In Texas, earthquakes have occurred in close association with petroleum activities since 1925, but the rate of earthquakes in the last decade has increased over tenfold \citep{Frohlich2016HistoricalReviewInduced, Skoumal2020InducedSeismicityDelaware}.
To better understand the causes of these earthquakes and to assess the likelihood of infrastructure damage and safety concerns, 
the State of Texas funded the Texas Seismological Network (TexNet) to record earthquakes down to M2.0 across the state since 2017 \citep{Savvaidis2019TexnetStatewideSeismological}. 
By that time, there were over 130,000 active production wells, 23,000 active EOR wells, and nearly 3800 active saltwater disposal (SWD) wells in the Permian Basin (Figure \ref{fig:permian-oil-6panel} (a)).  
The volumes of petroleum production (Figure \ref{fig:permian-oil-6panel} (b, c)) and wastewater injection (Figure \ref{fig:permian-oil-6panel} (e, f)) have been rising in many locations; however, the recently cataloged earthquakes are spatially clustered (Figure \ref{fig:permian-oil-6panel} (c)). One significant cluster is near Pecos, TX in the Delaware Basin, which experienced over 2000 earthquakes in 2017 \citep{Frohlich2019OnsetCauseIncreased}. 
%Subsequent activity has increased considerably, and the region experienced more than 2000 earthquakes in 2017. 


%TexNet seismic data will be most meaningful when combined with knowledge of the subsurface \citep{Council2013InducedSeismicityPotential, TheAcademyofMedicine2017EnvironmentalCommunityImpacts}; however, measuring the subsurface at a basin-scale is both expensive and technically challenging.

\begin{figure}[hbt!]
	\centering
	\includegraphics[width=0.99\linewidth]{figures/chapter4-grl/supplement/figureS1-cisr-data.pdf}
	\caption[Shale development and seismicity in the Permian Basin.]{Shale development and seismicity in the Permian Basin through 2018. 
		(a) Locations of oil production, EOR, and saltwater disposal (SWD) wells active in 2017. (b) Annual oil production volume on a 10-mile grid in 2017. (c) Permian Basin oil production rate as reported by the Texas Railroad Commission. (d) Locations of earthquake hypocenters detected by TexNet in 2017. The color and size of a circle indicates the estimated earthquake depth and magnitude. (e) Annual injection volume (including both SWD and EOR wells) on a 5-mile grid. (f) Permian region injection rate (including both SWD and EOR wells) as reported by the Texas Railroad Commission.
	}
	\label{fig:permian-oil-6panel}
\end{figure}





%
%%\begin{figure}[hbt!]
%\begin{figure}
%	\centering
%	%\includegraphics[width=0.9\linewidth]{figures/figure4-study-area.pdf}
%	\includegraphics[width=0.96\linewidth]{figures/chapter3-permian/figure-study-area.png}
%	\caption[Subbasins within the Permian Basin]{
%		The subbasins of the Permian Basin are the Delaware Basin to the west (green) and Midland Basin to the east (purple), which are separated by the uplifed Central Basin Platform (teal).
%	}
%	\label{fig:study-area-plain}
%\end{figure}
%


% Intuitively, the spatial and temporal correlation between human activity and event occurrence may represent a key parameter to discriminate between natural and anthropogenic seismicity, but unfortunately this is not always the case. Indeed, many industrial operations involving subsurface fluid injection (e.g., wastewaterdisposal) can transmit pore pressure changes at large distance, causing earthquakes several kilometers awayfrom the industrial site.


%Attributing causation of earthquakes to individual wells may be impossible in many cases due to the proximity of many wells, but 

Several studies have used spatio-temporal analyses to link certain instances of wastewater injection and hydraulic fracturing to earthquakes \citep{Savvaidis2020InducedSeismicityDelaware, Skoumal2020InducedSeismicityDelaware, Grigoratos2020EarthquakesInducedWastewatera}, but attributing causation of earthquakes to individual wells and discriminating induced from natural seismicity is extremely challenging \citep{Grigoli2017CurrentChallengesMonitoring, Dahm2012RecommendationDiscriminationHuman, Verdon2019ImprovedFrameworkDiscriminating,Frohlich2016HistoricalReviewInduced, Frohlich2016ReplyCommentA}.
%Without a detailed study involving multiple disciplines and observational datasets, 
%Recent efforts have considerably improved the public knowledge of subsurface fault locations (Horne?), geological characterization \citep{Smye2021LithologyReservoirProperties, Smye2021VariationsVerticalStress}, pore pressure distribution and... fault system stability \citep{Hennings2021StabilityFaultSystems} 
Understanding the nature and causes of earthquakes and how they are linked to certain production and disposal requires extensive knowledge of the subsurface.
However, subsurface measurements of pore pressure changes can be difficult or impossible to collect at a regional scale.


\section{Available Geodetic Data}
%- next: geodetic data
%- need basin wide character...
%- *check PNAS independent section*
%- first GPS: show one GPS time series
%- all look like this
%- 'for long time, assumed rigid, no deformation'
%- insar has this coverage
%- then talk processing
%- thern ' infollowing section, well see how to produce robust surface defo' (presumable revied all the challenges...)
%



\begin{figure}
	\centering
	\includegraphics[width=0.9\linewidth]{figures/chapter4-grl/figure1-study-area.pdf}
	\caption[GPS and InSAR data coverage over the Permian Basin.]{GPS and InSAR data coverage over the Permian Basin. Yellow dots indicate GPS permanent stations. Teal and red boxes indicate ascending path 78 and descending path 85 paths of Sentinel 1 InSAR coverage, respectively. 
%	Each path contains over 80 SAR acquisitions, leading to over 3500 interferograms per path at 120 m pixel spacing.
}
	\label{fig:paper1-study-area}
\end{figure}



%The coverage of GPS permanent stations in West Texas is sparse, and there are 14 permanent GPS stations that recorded daily east, north, and up (ENU) surface deformation measurements (Figure \ref{fig:paper1-study-area}) \citep{Blewitt2018HarnessingGpsData}. After removing the common tectonic motion, all GPS stations showed little surface deformation (0-3 mm/year) between Nov. 2014 and Jan. 2019.

The coverage of GPS permanent stations in West Texas is sparse, and there are no stations in the Delaware Basin. At 14 stations in the Midland Basin and the Central Basin Platform, daily east, north, and up (ENU) surface deformation measurements were processed by the Nevada Geodetic Laboratory \citep{Blewitt2018HarnessingGpsData} (Figure \ref{fig:paper1-study-area}). After removing the common tectonic motion, little motion (0-3 mm/year) was observed at all GPS stations over the study period (Figure \ref{fig:ch3-gps}). Because energy-related injection and extraction activities often occur within deep and rigid subsurface formations, it has been common to assume little deformation can be detected at Earth's surface. 
%Thus, the available permanent GPS stations are too sparse to provide a full picture of surface changes, 


InSAR surface deformation measurements have much broader spatial coverage, and they provide a key observable to fill the gaps left by GPS. They allow us to estimate locations of pressure build up from fluid injection, barriers to subsurface fluid flow, and unmapped faults. However, creating accurate maps of surface deformation using InSAR can be very challenging at the scale of the full Permian Basin.


\begin{figure}
	\centering
	\includegraphics[width=0.9\linewidth]{figures/chapter3-permian/gps-txmc.pdf}
	\caption[Example permanent GPS station measurements]{
		Example measurements of east, north, and vertical components of surface deformation from the permanent GPS station TXMC (Figure \ref{fig:paper1-study-area}). All stations indicated in (Figure \ref{fig:paper1-study-area}) show similarly small deformation during the 2015-2019 period.
	}
	\label{fig:ch3-gps}
\end{figure}


\section{InSAR processing strategy}
\label{sec:ch2-insar-processing}

Using a geocoded SLC processor \citep{Zheng2017PhaseCorrectionSingle, Zebker2017UserFriendlyInsar} (Section \ref{sec:ch2-processing}), we processed 91 ascending (path 78, frames 94-104) and 82 descending (path 85, frames 483-493) Sentinel-1 scenes acquired between Nov. 2014 and Jan. 2019 (Figure \ref{fig:paper1-study-area}). We generated more than 7000 interferograms with 120 meter pixel spacing and a maximum temporal baseline of 800 days. No spatial baseline threshold was imposed in the interferogram formation. Because few decorrelation artifacts were presented, we were able to unwrap all interferograms without additional spatial filtering using the Statistical-cost, Network-flow Algorithm for Phase Unwrapping (SNAPHU) \citep{Chen2001TwoDimensionalPhase}. We removed long-wavelength phase ramps due to long-wavelength tropospheric noise, using a planar phase model. Comparable interferograms can be generated using other processors such as the InSAR Scientific Computing Environment (ISCE) \citep{Rosen2012InsarScientificComputing}. 
We chose the GPS station TXKM as the reference point for both ascending and descending InSAR data, and we used the remaining 13 stations as controls to assess InSAR measurement uncertainty as described in Chapter \ref{CHAP:4-GRL}.

As outlined in Section \ref{sec:ch2-noise}, interferograms may include noise from many possible sources. %orbital errors, phase decorrelation, unwrapping errors, DEM inaccuracies, ionospheric and tropospheric artifacts, and other residual noise terms. 
In our West Texas data, interferograms showing signs of $\Delta \phi_{orb}$, $\Delta \phi_{decor}$, and $\Delta \phi_{unwrap}$ were either removed or were negligible. Because the Permian Basin is located in the middle latitudes and is relatively flat, $\Delta \phi_{iono}$ and $\Delta \phi_{dem}$  are not substantial \citep{Fattahi2013DemErrorCorrection, Liang2019IonosphericCorrectionInsar}. 
%For the reminder of this section, we focused on evaluating and mitigating the impact of tropospheric noise on the West Texas Sentinel-1 data.
As described in Section \ref{sec:ch2-noise-tropo}, tropospheric noise $\Delta \phi_{tropo}$ consists of a stratified component that correlates with topography \citep{Doin2009CorrectionsStratifiedTropospheric} and a turbulent component that is random at time scales longer than one day \citep{Emardson2003NeutralAtmosphericDelay}. We checked for a stratified tropospheric noise component using the Generic Atmospheric Correction Online Service (GACOS), and we found that the stratified tropospheric errors in our Sentinel-1 data are minimal.
%This is because the oil-producing region of the Permian Basin is relatively flat, and we observed little correlation between InSAR LOS measurements and the Digital Elevation Model (DEM) data 
%(see Section \ref{sec:ch3-noise-tropo-mitigate}).
The dominant noise term for our West Texas Sentinel-1 data was tropospheric noise.
In Chapters \ref{CHAP:4-GRL} and \ref{CHAP:5-robust-ts}, we present evaluation mitigation strategies capable of producing robust estimates of surface deformation.

%	- thern ' infollowing section, well see how to produce robust surface defo' (presumable revied all the challenges...)


%In this thesis, we develop techniques for mitigating strong tropospheric noise to produce reliable surface deformation maps over large regions. These techniques are applicable to any large-scale deformation mapping, but they are critical to successfully detecting the centimeter-level changes occurring in the Permian Basin that are masked by 10s of centimeters of tropospheric noise. 


%Understanding the nature and causes of earthquakes and how they are linked to certain production and disposal wells requires extensive knowledge of the subsurface. InSAR surface deformation measurements allow us to estimate the distribution of fault slip at depth and infer the associated seismic risk \citep{Segall2010EarthquakeVolcanoDeformation, Huang2017FaultGeometryInversion}. Furthermore, measurements of reservoir inflation due to wastewater injection allows operators to assess pressure build-up in disposal aquifers and the associated triggered seismicity risk. It is also worth noting that oil recovery in shale wells is notoriously low \citep{Clark2009DeterminationRecoveryFactor}, and the performance of these wells varies significantly. InSAR can be employed to map subsurface fluid depletion and pressurization. These surface deformation observations, when coupled with reservoir compaction inversion modeling, can be used to assess the areal effectiveness of oil and gas extraction operations \citep{Du2001PoroelasticReservoirModel, Vasco2005UseQuasiStatic}. 



%%PNAS
%
%\section*{Shale development and induced seismicity}
%
%\begin{figure*}[hbt!]
%	\centering
%	\includegraphics[width=0.99\linewidth]{figures/Figure1.pdf}
%	\caption{Shale development and induced seismicity in the Permian Basin. (a) Locations of oil production, enhanced oil recovery (EOR), and saltwater disposal (SWD) wells activ  e in 2017. (b) Annual oil production volume on a 10-mile grid in 2017. (c) Permian region oil production rate as reported by the Texas Railroad Commission. (d) Locations of e  arthquake hypocenters detected by TexNet in 2017. The color and size of a circle indicates the estimated earthquake depth and magnitude. (e) Annual injection volume (includin  g both SWD and EOR wells) on a 5-mile grid. (f) Permian region injection rate (including both SWD and EOR wells) as reported by the Texas Railroad Commission.
%	}
%	\label{fig:Permian}
%\end{figure*}
%
%\begin{figure}[hbt!]
%	\centering
%	\includegraphics[width=0.99\linewidth]{figures/Figure2.pdf}
%	\caption{GPS and InSAR data coverage over the Permian Basin. Yellow dots indicate GPS permanent stations. Teal and red boxes indicate ascending path 78 and descending path 85   paths of Sentinel 1 InSAR coverage, respectively. Each path contains over 80 SAR acquisitions, leading to over 3500 interferograms per path at 120 m pixel spacing.
%	}
%	--\label{fig:study-area}
%\end{figure}
%The success of shale development technologies \cite{Waters2006use} has opened up vast shale resources for economically viable oil and gas production. Based on a recent assess  ment by the U.S. Geological Survey (USGS), the Wolfcamp shale in Texas' Permian Basin is the largest continuous oil field that has ever been discovered in the United States \  cite{GaswirthAssessment2016}. As of 2017, there were over 130,000 active production wells, 23,041 active enhanced oil production (EOR) wells, and 3794 active saltwater dispos  al (SWD) wells in the Permian Basin (Figure \ref{fig:Permian} (a)). It has been recognized that injection or withdrawal of fluids from the subsurface can induce earthquakes a  long existing faults \cite{Ellsworth2013, simpson1988two}. While petroleum production and wastewater injection volumes have been rising throughout the basin, the recently cat  aloged earthquakes are spatially clustered (Figure \ref{fig:Permian} (b)-(f)). One significant cluster is near Pecos, TX, where increased seismic activity began in 2009. Subs  equent activity has increased considerably, with more than 2000 earthquakes identified in 2017 \cite{Frohlich2019}.
%
% Understanding the nature and causes of earthquakes and how they are linked to certain production and disposal wells requires extensive knowledge of the subsurface. InSAR surf  ace deformation measurements allow us to estimate the distribution of fault slip at depth and infer the associated seismic risk \cite{Segall2010, huang2017fault}. Furthermore  , measurements of reservoir inflation due to wastewater injection allows operators to assess pressure build-up in disposal aquifers and the associated triggered seismicity risk. It is also worth noting that oil recovery in shale wells is notoriously low \cite{clark2009determination}, and the performance of these wells varies significantly. InSAR can be employed to map subsurface fluid depletion and pressurization. These surface deformation observations, when coupled with reservoir compaction inversion modeling, can be used to assess the areal effectiveness of oil and gas extraction operations \cite{Du2001, Vasco2005}.



%%  \section*{Geodetic data in West Texas}
% The coverage of GPS permanent stations in West Texas is sparse (Figure \ref{fig:study-area}). There are no stations in the Delaware Basin. At 14 stations in the Midland Basin and the Central Basin Platform, daily east, north, and up (ENU) surface deformation measurements were processed by the Nevada Geodetic Laboratory \cite{Blewitt2018}. After removing the common tectonic motion, little motion was observed at all GPS stations over the study period.
% 
% We processed 91 ascending (path 78) and 82 descending (path 85) Sentinel-1 scenes acquired between Nov. 2014 and Jan. 2019 (Figure \ref{fig:study-area}). We generated more th  an 7000 interferograms with 120 meter pixel spacing and a maximum temporal baseline of 800 days. Given that few decorrelation artifacts are present in these interferograms du  e to the lack of vegetation, we did not impose any spatial baseline threshold. Each interferogram, formed using two radar images from the same path, measures surface deformat  ion between the Sentinel-1 acquisition dates along the radar line-of-sight (LOS) direction. We selected all interferograms that were generated using SAR acquisitions within a period of interest, and we estimated the average LOS deformation rate during this period by computing the ratio between the sum of all interferograms and the sum of their time spans \cite{Sandwell1998}. A comparable solution can be derived using the Small Baseline method with a linear deformation model \cite{SBAS2002}. In the region where the two paths overlap, we decomposed the eastward and vertical motion from the ascending and descending LOS solutions (\textit{Supporting Information}).
% 
% Because InSAR only measures relative motion between ground pixels, we chose the GPS station TXKM to serve as a common reference for deformation measurements. Given that little relative motion was observed at the remaining 13 GPS stations, we used these locations as controls. For example, at the GPS station TXMC, most interferograms indicate near   zero motion over their time spans, with no systematic bias and a standard deviation $\sigma$ of 7.2 cm (Figure \ref{fig:outliers} (a)). The spread of the LOS measurements is   likely caused by tropospheric turbulence that is approximately random at time scales longer than 1 day \citep{Emardson2003}, and local storms can produce $\pm$ 15 cm outliers   \cite{Bekaert2015statistical}. We observed similar noise distributions at all 13 GPS locations, and the tropospheric noise level increases linearly with the distance from th  e reference GPS station. At 40 km, $\sigma$ equals 3.7 cm. For every additional 50 km increase, $\sigma$ increases by 1.7 cm (Figure \ref{fig:outliers} (b)).
%
% Sentinel-1 satellites sample the Earth much more frequently in time compared to earlier InSAR missions. This allows us to average thousands of InSAR measurements and reduce the impact of random tropospheric noise \cite{Zebker1997}. However, the quality of InSAR measurements is heterogeneous. For example, local extreme weather events only impact certain pixels on a few SAR acquisition dates. Finding an automated way to identify these outlier measurements from thousands of interferograms that cover a wide region with very high spatial resolution is challenging, yet critical for further noise reduction.
%
%
