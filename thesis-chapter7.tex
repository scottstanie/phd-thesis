
\chapter{Summary and Conclusions}
\label{CHAP:7}

In this thesis, we demonstrated the ability to derive accurate surface deformation maps over broad areas from InSAR measurements corrupted by severe tropospheric noise. Specifically, we have made the following contributions:



\begin{enumerate}
	
	\item We developed Python-based InSAR time series analysis software to ingest geocoded SAR images, extract surface deformation, decompose multiple geometries into horizontal and vertical deformation, and verify results with permanent GPS stations.
	
	\item We performed a rigorous uncertainty analysis to identify the dominant noise source in the Permian Basin for Sentinel-1.  In the process, we developed a method to estimate the tropospheric noise and its power spectral density from InSAR data.
	
	\item We designed scalable, robust time series algorithms for tracking yearly changes of deformation over large regions.
	
	\item We created millimeter-level accurate maps of cumulative surface deformation that are available online for analysis. These maps contain many subsidence and uplift features near oil and gas production, as well as linear patterns near clusters of seismic activity.
	
	\item We created a computer vision algorithm to automatically detect surface deformation signals of unknown sizes in large InSAR maps. The detection algorithm produces uncertainty estimates for each visual feature in the InSAR data.
	
		
\end{enumerate}

With multiple government and commerical SAR missions scheduled in upcoming years, the ability to leverage...

Several efforts are underway to leverage this data growth to produce standardized surface deformation products, including the 
with scalable algorithm will become increasingly important.  

Specficially, the Observational Products for End-Users from Remote Sensing Analysis (OPERA) project at NASA JPL has been tasked to produce a North America land-surface displacement product from Sentinel-1 and NISAR data
\citep{Bekaert2021IntroducingOperaProject}.The techniques developed in this thesis can be leveraged and extended to produce and analyze large-scale deformation maps. 
Specific future work in this direction may include:

\begin{itemize}

%Future conclusions- 
%ch5, seasonal lack. Maybe something about large scale filtering for knowing what you can get rid of? From modeling, converting defo to pore pressures, determining east motion. 
% ch5: modeling causes. proportion of aseismic slip (cite karissa,eyre,)
%Ch 6-  1. Turb is dominant. 2. No stratified. 3. Decorr to further resample and get uncertainty. 4. Resampling in a way to mimic the seasonal variations, rather than mean. 

%\item Geomechanical modeling of the time-varying deformation maps over the Permian Basin. As Figure \ref{fig:ch5-discuss-eqs} suggests, 

\item Combining subsurface pore pressure models using the wastewater disposal history (e.g. \cite{Ge2022RecentWaterDisposal}) with the recent changes to surface deformation.

\item Using the vertical and eastward maps of Chapter \ref{CHAP:5-robust-ts} to estimate locations of new, unmapped faults \citep{Horne2021BasementRootedFaults}.

% From Agnew, 1992
%it should be noted that the PSD description is not in general a complete specification of a process; the power spectrum is only a summary of second moments (variance versus frequency). Stochastic processes with identical power spectra can have very different appearances in the time domain


\item Accounting for both seasonal and stratified tropospheric noise, in addition to tropospheric turbulence used in Chapter \ref{CHAP:6-blob}, to more reliably map of mountainous study areas. Additionally, the seasonal variations of tropospheric noise may be important to simulate to accurately infer uncertainty for other time series analysis methods.

\item Incorporating additional error sources into the noise simulations presented in Chapter \ref{CHAP:6-blob}, including unwrapping errors \citep{Yunjun2019SmallBaselineInsar}, closure phase bias \citep{Zheng2022ClosurePhaseSystematic}, and speckle-induced uncertainty \citep{Zwieback2022ReliableInsarPhase}. 
Since difference study regions experience  a difference balance of phase errors from Equation \eqref{eq:ch2-insar-noise-terms} which need to be included for a reliable uncertainty estimate in the final time series.

\item Combining auxiliary data sources with the deformation detections of Chapter \ref{CHAP:6-blob} to classify/categorize possible causes of deformation. This data fusion may become increasingly useful as larger deformation maps are accessible through cloud computing environments (e.g. \cite{Kellndorfer2022GlobalSeasonalSentinel}).

% Estimation of errors from decorr, from fading signal bias (yujies paper?), from aliasing effect (cite karissa) and phase unwrapping errors (cite yunjun), and from speckle induced errors (Zwiebeck)

\end{itemize}

