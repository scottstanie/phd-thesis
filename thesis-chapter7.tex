
\chapter{Summary and Conclusions}
\label{CHAP:7}

In this dissertation, we demonstrated the ability to derive accurate surface deformation maps over broad areas from InSAR measurements corrupted by severe tropospheric noise. We further demonstrated techniques for estimating uncertainty of automatically detected deformation features. Specifically, we have made the following contributions:


\begin{enumerate}
	\item We developed Python-based InSAR time series analysis software that processes geocoded SAR images acquired from multiple imaging geometries and reconstructs surface deformation in eastward and vertical directions.
	
	\item We performed a rigorous analysis of all noise sources in the Permian Basin Sentinel-1 InSAR data. We identified that the dominant noise term is the tropospheric turbulence noise with up to 15 cm non-Gaussian outliers. We developed methods for characterizing tropospheric noise and its power spectral density directly from InSAR data, as well as methods for mitigating the impact of the troposphere noise outliers.
	
	\item We designed scalable, robust time series algorithms for reconstructing the temporal evolution of surface deformation over very wide regions. Based on independent validation from GPS permanent stations, we achieved millimeter-level accuracy in the cumulative surface deformation solutions.
	
	\item We developed a computer vision algorithm for automatically detecting surface deformation signals of unknown sizes in basin-scale InSAR maps. The detection algorithm produces uncertainty measures for each detected feature based on a realistic tropospheric turbulence noise model.
	
	\item InSAR reveals numerous subsidence and uplift features near active production and disposal wells, as well as linear deformation patterns associated with fault activities near clusters of seismic activity. Our InSAR deformation maps are now openly available through the Center for Integrated Seismicity Research (CISR) for the broader scientific community and stakeholders. 
	
		
\end{enumerate}

With multiple government and commercial SAR missions scheduled in upcoming years, there will be opportunities for creating continental and global deformation products from InSAR data.
For example, the Observational Products for End-Users from Remote Sensing Analysis (OPERA) project at NASA JPL has been tasked to produce a North America land-surface displacement product from Sentinel-1 and NISAR data
\citep{Bekaert2021IntroducingOperaProject}.
%The processing and analysis techniques 
Since the processing and analysis techniques required for such a product must be scalable and robust, the techniques developed here can contribute to these efforts.

Future work beyond extending the results presented here may include:

\begin{itemize}

%Future conclusions- 
%ch5, seasonal lack. Maybe something about large scale filtering for knowing what you can get rid of? From modeling, converting defo to pore pressures, determining east motion. 
% ch5: modeling causes. proportion of aseismic slip (cite karissa,eyre,)
%Ch 6-  1. Turb is dominant. 2. No stratified. 3. Decorr to further resample and get uncertainty. 4. Resampling in a way to mimic the seasonal variations, rather than mean. 

%\item Geomechanical modeling of the time-varying deformation maps over the Permian Basin. As Figure \ref{fig:ch5-discuss-eqs} suggests, 

\item Combining subsurface pore pressure models using the wastewater disposal history (e.g. \cite{Ge2022RecentWaterDisposal}) with the recent changes to surface deformation to determine subsurface flow barriers and locations of elevated pressure. This effort could leverage the recent detailed compilation of known faults in the Delaware Basin \citep{Horne2021BasementRootedFaults}.

% From Agnew, 1992
%it should be noted that the PSD description is not in general a complete specification of a process; the power spectrum is only a summary of second moments (variance versus frequency). Stochastic processes with identical power spectra can have very different appearances in the time domain


\item Simulating both seasonal and stratified tropospheric noise to more accurately estimate uncertainty. The noise simulations presented in Chapter \ref{CHAP:6-blob} used tropospheric turbulence, which was the dominant error source for West Texas. The seasonal variations of tropospheric noise may be important to accurately infer uncertainty for other study areas.
% or when using other time series analysis methods.

\item Incorporating additional error sources into the noise simulations, including unwrapping errors \citep{Yunjun2019SmallBaselineInsar}, closure phase bias \citep{Zheng2022ClosurePhaseSystematic}, and speckle-induced uncertainty \citep{Zwieback2022ReliableInsarPhase}. 
%Since difference study regions experience  a difference balance of phase errors from Equation \eqref{eq:ch2-insar-noise-terms} which need to be included for a reliable uncertainty estimate in the final time series.

\item Combining auxiliary data sources with the automatically detected deformation features of Chapter \ref{CHAP:6-blob} to classify/categorize possible causes of deformation.

\item Increasing accessibility of InSAR-derived products by leveraging new, cloud-friendly data formats (e.g. \cite{Kellndorfer2022GlobalSeasonalSentinel}).

% Estimation of errors from decorr, from fading signal bias (yujies paper?), from aliasing effect (cite karissa) and phase unwrapping errors (cite yunjun), and from speckle induced errors (Zwiebeck)

\end{itemize}

