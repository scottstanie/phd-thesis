


\chapter{Robust Time Series Methods for Long-term Nonlinear Deformation}
\label{CHAP:5-robust-ts}


In this 

Section \ref{sec:ch4-method-compare} compared the stacking method used in Chapter \ref{CHAP:4-GRL} with several alternatives .

- Problem: 
1. Longer time series, the linear assumption becomes less valid.  This is at odds with the balance of using more measurements to average out tropospheric noise
2. Methods incorporating temporal smoothness constraints (Find a few. e.g. Karissa? other temp smooth solvers...) or temp lienar filtering methods get influenced by strong tropo days. Even with smoothing, there are anomalous bumps 
	- can get interpreted as deformation if we assume that all movements remaining after filtering are due to the low-pass deformation
	

\section{Methods}

\section{Synthetic Example}

\section{Results}

\subsection{Permian Basin 7-year Time Series}

\subsection{GPS Comparison}

\section{Discussion}

%1. Limitations of stacking to long time series
%2. Robust Time Series Methods
   %1. Regularization (Supplement from GRL)
   %2. LOWESS smoothing
%3. Synthetic Example
%4. 7 Year Time Series for the Permian Basin
%1. Comparison to GPS
%2. Anthropogenic Caused Deformation Patterns


%Problems with pixelwise uq
%- Image of blob, with 8 mm cutoff, question which part you trust and not
%- Leads into feature-wise uq

