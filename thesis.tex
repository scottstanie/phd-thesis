%% utexasthesis.cls is available from https://github.com/linguistics/utexas-latex
\documentclass{utexasthesis}
% \documentclass[copyright,12pt,onehalfspacing,draft]{utexasthesis}

%% Required fields
%% ===============
%% Full official title of your thesis (use \\ to force a line break)
\title{Computer Vision for Radar Interferometry \\ Over West Texas }
%% Your full official name
\author{Scott Staniewicz}
%% Month and year of graduation (month may be May, August, or December)
\graduationdate{May}{2022}
%% Your thesis supervisor, full name only
\supervisor{Jingyi Ann Chen}
%% Your thesis co-supervisor, if any (leave commented if not applicable)
% \cosupervisor{Cosupervisor Name}
%% Other committee members full names, comma-separated.
%% Comment this out if empty, e.g., for a masters thesis with only a supervisor and cosupervisor.
\othercommitteemembers{Srinivas Bettadpur, Todd Humphreys, Jon Olson}

%% Optional customizations
%% =======================
%% Use Palatino as the primary font face
\usepackage{palatino}
\usepackage{amsmath}
\usepackage{bm}
\usepackage{minted}
%% and Computer Modern Typewriter Proportional as the teletype font face
\renewcommand*\ttdefault{cmvtt}

\begin{document}

%% This produces the copyright page (if specified), the signature page, and title page.
\maketitle

%% The dedication is optional and fills an entire page.
\begin{dedication}
  I dedicate this to Moxie
\end{dedication}

%% The acknowledgments, abstract, and table(s) of contents/tables/figures pages are numbered with roman numerals.

%% The acknowledgments is optional and fills an entire page.
\begin{acknowledgments}
  Many thanks to Moxie
\end{acknowledgments}

%% The abstract is required.
\begin{abstract}
  Indent and begin abstract here. It should be a concise statement of the nature and content of the ETD. The text must be either double-spaced or 1.5-spaced. Abstracts should be limited to 350 words.
\end{abstract}

%% The table of contents is required.
\maketableofcontents

%% The following pages are numbered with arabic numerals, starting with 1

\chapter{Introduction}

Your ETD must be correct in spelling and punctuation and presented in a consistent, structured format. A single, legible font must be used throughout, the only exceptions being in tables, figures, graphs, appendices, and supplemental files. Headings may be bolded and no more than 2 points larger than the rest of the text. The font size should be sufficient for the average person to read the document on a computer monitor without difficulty (12-pt is recommended.) Accuracy and consistency in presentation and form make your ETD a usable research tool for other readers.


\section{A Section}

This is the first line of the first section of the first chapter, but not the first line of the first chapter, which belongs to no section.


\chapter{Atmospheric Noise Analysis and Simulation}
\label{chap:atmo-noise}

This is the first line of Chapter \ref{chap:atmo-noise}, 

\section{Disambiguating terms from fractal surfaces, Gaussian random
fields, and power
laws}
\label{disambiguating-terms-from-fractal-surfaces-gaussian-random-fields-and-power-laws}

Part that I wanted to write down:

Another way to consider fractals: If you generate white or pink noise, unlike human speech or music, it will sound the same at any speed.

Note on Brownian motion being an integral of white noise.
For any complex sinusoid $x(t) = e^{j2 \pi f t}$ of frequency $f$, the effect in the frequency domain can found by noting that 

\begin{equation}
\int x(t) \, dt = \frac{1}{j 2 \pi f} e^{j2\pi f t} = \frac{1}{j 2 \pi f} x(t)
\end{equation}

Since integration is linear and time invariant, we can think of it as an LTI filter $h$ with frequency response $H_i(f) = \frac{1}{j 2 \pi f} $, which has a $|1/f|$ magnitude response, or $|1/f^2|$ in power.

A white noise process $w(t)$ can be defined as having a flat power spectral density (PSD) $S_W(f) = \sigma^2$. Integrating white noise results in the PSD being shaped into $S_W(f) H_i(f) = \sigma / f^2$.

\textbf{Driving question}:

Is the fractal atmospheric surface (a type of spatially correlated
noise), which has a power-law power spectrum of

\[S(f) = (1/f)^{\beta}\] also a Gaussian random field?

The three related subjects are\ldots{} - Gaussian Fields - Fractal
surfaces - Power-law power spectra Can all three be present? Or are some
mutually exclusive?

\textbf{POSSIBLE SUMMARY ANSWER}

Yes, the surface can be 1. fractal, since the structure function (aka
semi-variogram) of the atmospheric delay has a power-law form (Hanssen,
Eq 4.7.10) \footnote{Hanssen, Ramon F. Radar interferometry: data
  interpretation and error analysis. Vol. 2. Springer Science \&
  Business Media, 2001.} :

\[D_s(\rho) \sim \rho^{5/3}\]

\begin{enumerate}

\item
  A Gaussian field, since by (Cressie, 1993) Eq. 5.5.1 \footnote{Cressie,
    Noel. Statistics for spatial data. John Wiley \& Sons, 2015.},
\end{enumerate}

More generally, a fractional Brownian motion in \(R^d\) is a Gaussian
process \(Z(\cdot)\) characterized by a covariance function of the form
with

\[C(s, t) = \frac{1}{2}\left(s^{2H} + t^{2H} + |s-t|^{2H}\right)\]

and a variogram of the form

\[E[(Z(s + h) - Z(s))^2] \propto ||h||^{2H}\] So, for the Hanssen
structure fucntion, \(H=5/6\)

\begin{enumerate}
\def\labelenumi{\arabic{enumi}.}
\setcounter{enumi}{2}

\item
  A signal power law power spectrum \(S(k)\) of the form (Hanssen Eq.
  4.7.12)
\end{enumerate}

\[P(k) = P_0 \left(f/f_0\right)^{-\beta}\] where \(P_0\) is the power at
reference frequency \(f_0\) and \(\beta\) is called the \emph{spectral
index}. The structure function of a signal with this power spectrum can
be written, using the same \(\beta\), as
\[D_{\phi}(\rho) \propto \rho^{\beta-1}\] which means that if the
structure function exponent is \(5/3\), then \(\beta=8/3\) for the
spectrum slope.

My confusion earlier: thinking that the frequency content of the signal
was doing to dictate the amplitude distribution in space (or time for
1D)\ldots{} In the same way that noise can be white (a
frequency/spectrum description) and either uniform or Gaussian (a time
or space description), the random process can be a Gaussian process, but
have the same power-law exponent.

LAST QUESTIONs: - BIGGEST UNKNOWN: Hanssen says the structure
functions/SMV is \(\rho^{5/3}\), which means in never levels off, and he
says the covariance function doesn't exist\ldots{} - Does this imply
that it can't be a Gaussian field? - \textbf{ANSWER}: most places seem
to say that fractional brownian motion is still a Gaussian field
(Brownian sheet having covariance = min(s, t)) - \textbf{TODO} This
means it has stationary increments, but isn't 2nd order
stationary\ldots{} I thought other places said ``Gaussian random fields
are 2nd order stationary (and sometimes strictly stationary under some
condition)'' - or does out limit study area + ramp removal mean in
practice it levels off\ldots{} - if a random field is Gaussian, does
it's structure func/SMV have to be gaussian?\ldots{}. NO. thats related
to the Covariance FUNCTION\ldots{} aka, the covariance of the Gaussian
variables at a given distance apart! the further away, the different the
covariance is\ldots{} but the entire process is still only characterized
by it's mean (assumed 0), and covariance function (related to the
structure function) - SO, a power-law SMV is fine for a Gaussian
process. and it will have a power law power-spectrum by FT:
https://math.stackexchange.com/questions/2173780/computing-fourier-transform-of-power-law

\begin{itemize}

\item
  I thought that Gaussian processes had have Gaussian Fourier
  transforms\ldots{}

  \begin{itemize}
  
  \item
    answer? the power spectrum relates to the FT of the covariance
    function of a (second order stationary process) (or, with different
    terms, ACF <->PSD)
  \end{itemize}
\item
  so i think ``fractal'' descriptions are orthogonal to both ``Gaussian
  process'' and ``Power spectrum'' shape\ldots{} they might just be 3
  completely separate axes, where none imply others
\end{itemize}

\hypertarget{definitions}{%
\section{Definitions}\label{definitions}}

From \cite{Hanssen2001} this, I don't see \citet{Hanssen2001} what the difference is \citep{Hanssen2001} , 4.7 intro: 

\begin{quote}
    
The behavior of atmospheric signal in radar interferograms can be mathematically described using several interrelated measures such as the power spectrum, the covariance function, the structure function, and the fractal dimension.

The power spectrum is useful to recognize scaling properties of the data or to distinguish different scaling regimes.
\end{quote}

TODO 1. Random field 1. Gaussian random field 1. Power spectrum


\subsection{Conclusions}



%% Insert the bibliography.
%% The style file, i.e., 'name.bst' for \bibliographystyle{name},
%% can be any name.bst available in your TeX distribution.
\bibliographystyle{plainnat}
\makebibliography{insar,permian,detection,imagestuff,statistics}

%% The Vita is optional, but must take no more than a single page if included.
\begin{vita}
  Full Official Name was born in Austin, Texas. After completing their work and graduating from Austin High School, they went to college somewhere, and graduated, but then decided two graduations weren't enough.
  After that, they entered the Graduate School at the University of Texas at Austin.

  %% The graduate school recommends using an email address as your address.
  \begin{address}
    scott.stanie@utexas.edu

    123 Main St.

    Austin, Texas 78712
  \end{address}

  %% declaring a typist is optional
  \declaretypist{the author}
\end{vita}

\end{document}
